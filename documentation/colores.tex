\section{Teor'ia de colores}
Antes de hablar sobre la teor'ia del color, recordemos que se hacia una critica sobre las herramientas generadoras de datos la deficiencia que tienen, el no tener el orden en que las tablas deben ser configuradas, para esto en este proyecto se hace uso de los colores tomando como indicadores visuales.
Existen varias teor'ias del color en este proyecto haremos uso de la \textit{teor'ia del color Kueppers}
\subsection{Qu'e es el Color realmente}
\begin{figure}[hbtp]
\centering
\includegraphics[scale=1]{images/farb-empf.png}
\caption{La cadena de efectos entre la Luz y la Sensaci'on del Color}
\end{figure}

Luz de Iluminaci'on(1) que cae sobre un objeto. Una parte de la luz es absorbida y cuando es tragada, aumenta la temperatura del objeto (2). La parte no absorbida de la Luz, llamamos Resto de Luz, que es reflejada como est'imulo (3) a los ojos del observador (4). Despu'es el ajuste del 'organo de la vista como medio de Adaptaci'on a la Intensidad, al Color de la Luz y al Contraste Simult'aneo, y donde es producido para cada punto de la retina un c'odigo el'ectrico. Ello es enviado por los nervios (5) al cerebro. De estos datos sin color est'a construido la imagen multicolor y tridimensional que ve el observador (6).
   