%% Los cap'itulos inician con \chapter{T'itulo}, estos aparecen numerados y
%% se incluyen en el 'indice general.
%%
%% Recuerda que aqu'i ya puedes escribir acentos como: 'a, 'e, 'i, etc.
%% La letra n con tilde es: 'n.

\chapter{Introducci'on}
\pagenumbering{arabic}
\renewcommand{\footrulewidth}{0.4pt}% linea footer
\pagestyle{plain}

Cuando se est\'a desarrollando software que hacen  uso de una bases de datos, que est\'a en gran parte de las aplicaciones. Es com\'un tener la necesidad de realizar pruebas sobre la base de datos, para verificar si los datos se est\'an gestionando de la manera correcta o verificar la eficiencia  de las consultas, para esto se insertan datos de prueba haciendo uso de comandos del Lenguaje de definici\'on de datos \texttt{(DDL)} \cite{ddl}.Otra manera es hacerlo por una interfaz gr\'afica dependiendo del sistema gestor de base de datos, entre las mas usadas se tiene a:
\begin{itemize}
\item \textbf{Postgresql}: PgAdmin.
\item \textbf{MySQL}: PhpMyAdmin, MySQL Workbench.
\item \textbf{SQLServer}: SQLServer managment studio.
\end{itemize}
Los que trabajan con base de datos llenaron datos de prueba de esta manera en alg\'un momento o siguen con los mismos m\'etodos.

Insertar datos de prueba en una base de datos de forma manual sea por interfaz gr\'afica o por una interfaz de texto como ser la consola, no suele ser tan agradable porque: conlleva mucho tiempo de trabajo; no siempre se tiene en mente lo que se quiere etc. Por lo que normalmente se insertan pocos datos de prueba para ver el comportamiento de las consultas que se realizan en el software. 

Una aplicaci\'on en un entorno de producci\'on de seguro no lleva pocos datos, al contrario llegan almacenar gigabytes de informaci'on.

El problema surge ah\'i, porque no es lo mismo hacer pruebas con cien datos que con cien mil datos o mas, si bien una consulta funciona de manera correcta con los pocos datos, es diferente el comportamiento con una poblaci'on de datos mas grande, puede que las consultas no funcionen de forma correcta o esta son muy lentas. Un error com'un en principiantes es realizar un \texttt{SELECT * FROM tabla} que de seguro no tendr\'ia problemas con los pocos datos, sin embargo en el entorno de producci'on es catastr\'ofico. 

Sin duda es un problema a considerarse en el desarrollo de software, sobre todo cuando no se realizan pruebas a una base de datos, dej\'andonos incertidumbres sobre su comportamiento con cantidades de datos que podr\'ia tener cuando el software ya est\'e en un entorno producci\'on.
\section{Aplicaciones generadores de datos de prueba para base de datos}
Para los que son nuevos en el mundo de base de datos es normal desconocer sobre herramientas que facilitan la tarea, y optan en llenar de forma manual una base de datos, que resulta ser tedioso y puede llegar a no gustar ser responsable de la parte de la base de datos, si a principios de la aparici'on de la base de datos no se contaba con herramientas de apoyo en esta \'area, hoy en d\'ia existen herramientas que automatizan procesos como el llenado de datos de prueba en cantidades grandes sobre una base de datos ya existente. 

Entre las herramientas para la generaci\'on de datos de prueba para base de datos, se puede mencionar a Datanamic Data Generator MultiDB perteneciente a Datanamic \cite{datanamic}, Generatedata para mysql de c\'odigo abierto con licencia GNU \cite{generatedata}, EMS Data Generator for MySQL de la l\'inea de EMS y EMS Data Generator for PostgreSQL tambi\'en de la l\'inea de EMS \cite{emsdatagenerator} y MyDatagen \cite{mydatagen}, son algunas que se puede mencionar lo cual no significa que sean las \'unicas, con estas herramientas es posible realizar el poblado de datos para su posterior realizaci\'on de pruebas.

De las herramientas el mas destacado es Datanamic, por la forma en como permite configurarlo. En la p\'agina oficial \cite{datanamic} estan disponibles para MySql, PostgreSQL, Oracle y otros. En el caso de Datanamic para PostgreSQL es una de las  m\'as destacables entre las mencionadas ,donde se puede ver opciones de generar datos lo primero es escoger una base de datos existente en otra base de datos y lo que hace el software es reconocer caracter\'isticas de la base de datos, con sus respectivos tipo de datos y por defecto presenta de generar cien registros, con la libertad de configurar a gusto adem\'as da la opci\'on de escoger la fuente de datos que se har\'a uso, a partir de una lista, otro es obtener datos como nombres por ejemplo a partir de un archivo, y poder seleccionar y escoger si ser\'a aleatoriamente o una forma secuencial, si ser\'a \'unico o si se repetir\'a esto dependiendo de c\'omo se quiere llenar datos.

De las otras herramientas mencionadas tienen una similitud en el manejo y en c\'omo se realiza el llenado de los datos, a diferencia de Generatedata que es una herramienta libre y de c\'odigo abierto escrita en JavaScript, PHP y MySQL. Que permite generar de una forma r'apida grandes vol\'umenes de datos personalizados en una variedad de formatos, para su uso en pruebas de software, rellenar bases de datos, etc. Los desarrolladores pueden escribir sus propios tipos de datos para generar nuevos tipos de datos aleatorios e incluso personalizar los tipos de exportaci\'on, Para las personas interesadas en la generaci\'on de datos de localizaci\'on geogr\'afica, se pueden a\~nadir nuevos complementos para proporcionar nombres de regiones (estados, provincias, territorios, etc.), nombres de ciudades y formatos de c\'odigos postales para su pa\'is, todo esto porque es libre de c\'odigo abierto, donde se puede observar que los datos a llenar a una base de datos los extrae de su propia base de datos que incluye Generatedata y en su modelo haciendo ingenier'ia reversa se puede visualizar que los datos maneja almacenado todos los datos posibles como el nombre de pa\'ises. Las cr\'iticas que podr\'ia tener esta herramienta es porque no hace el llenado a una base de datos existente lo cual lo quita puntos a favor adem\'as que solo funciona para MySQL entre sus punto a favor es que es libre de c\'odigo abierto. Sin embargo Datanamic viene para distintos motores de base de datos pero si hay uno que es multifuncional.

Con un an\'alisis sobre el uso de las herramientas de cualquiera de las mencionadas, llenando datos de prueba en una base de datos, el tiempo del llenado es considerablemente inferior a lo que llevar\'ia hacerlo manualmente, cuanto m\'as datos mayor es el tiempo en llenarlo en la forma manual, sin embargo haciendo uso de alguna de estas herramientas que ayudan en realizar esta tarea por nosotros, el tiempo es muy similar que llenar pocos registros as'i que es muy conveniente llenar la mayor cantidad de datos en la base de datos siempre que se disponga una fuente por ejemplo una lista de nombres, si se quiere llenar nombres en una entidad. 

Ser'ia mucho mejor encontrar informaci'on sobre c'omo llenar ya que hay poca informaci'on sobre estas herramientas con una documentaci'on no muy clara de algunas como MyDatagen, PgDatagen que sin embargo Datanamic si cuenta con una documnetacion mas clara \cite{datanamictutorial} de c\'omo se realiza el llenado. 

Una de las deficiencias que se puede encontrar y que le quita puntos a su favor, es cuando se tiene relaciones compuestas no las reconoce como tal y esto llega a ser un problema.

Las herramientas comerciales como es el caso de Datanamic, MyDatagen y PgDatagen, no provee el acceso al c'odigo fuente y es un problema saber c'omo es la logica de la generaci'on de datos, pero es observable mediante la interfaz gr'afica el como elige el generador de datos adecuado para cada columna, basado en las caracter'isticas de la columna, con m'as de cuarenta generadores de datos incorporados (espec'ificos para pa'ises e idiomas), genera datos realistas con el uso inteligente del generador (para, por ejemplo, c'odigos postales) y una gran colecci'on de listas con nombres, direcciones, ciudades, calles, etc. Obtiene datos al azar de una fuente externa y obtiene datos de fuentes existentes como las otras tablas, opci'on para deshabilitar los desencadenantes, vista previa en tiempo real de los datos que se van a generar, genera datos de prueba para una base de datos completa o para una selecci'on de tablas, incluye una utilidad de l'inea de comandos para automatizar a'un m'as el proceso de generaci'on de datos, inserta datos directamente en la base de datos o genera un secuencia de comandos \texttt{SQL} con instrucciones de inserci'on, guarda tu plan de generaci'on de datos de prueba a un archivo de proyecto, validación extensa de configuraci\'on del generador de datos, ejecuta secuencia de comandos de datos previas posteriores a la generaci'on, detecci\'on autom'atica de los cambios en el esquema de base de datos, opci'on de generar valores 'unicos.

De todas estas caracter'isticas es interesante conocer acerca de c'omo realiza la generaci'on de datos, al tratar de ver como lo realiza, no se cuenta con informaci'on suficiente de parte de la aplicaci'on, solo se puede visualizar. Sin embargo el objetivo es entender c'omo hace la generaci'on de datos seg'un lo requerido, al observar las herramientas listadas, los datos que generan van dependiendo siempre del tipo de dato, y poder tomar como par'ametros la cantidad de datos si ser'an 'unicos entre otras, todas estas caracteristicas da una idea de como hacer un generador de datos.

Al hacer uso de cualquiera de estas herramientas se puede encontrar con un problema que se considerar'ia que es una deficiencia, ninguna de las mencionadas trata de ayudar al usuario en el orden del llenado de las tablas, si bien internamente lo haga no muestra cuales son tablas que se deben configurar primero y cuales las siguientes as'i sucesivamente.