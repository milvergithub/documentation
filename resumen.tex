\chapter*{Resumen}
\thispagestyle{empty}
\renewcommand{\footrulewidth}{0.0pt}% linea footer
En el desarrollo de software que implica el uso de base de datos, es bastante com\'un realizar pruebas para conocer el comportamiento del software conjuntamente con la base de datos, para lo cual es necesario tener la base de datos de prueba similar a una base de datos en un entorno de producci\'on. 

Si bien se tiene claro que es necesario tener la base de datos con datos de prueba, no resulta ser tan sencillo poblar con datos de prueba, ya que como resulta ser tedioso, debido a que no siempre se tiene en mente lo que se quiere, tener en mente nombres de las tablas y quien referencia a quien, por lo cual normalmente se acostumbra a insertar pocos datos de prueba, como resultado deja la incertidumbre de como vaya a ser el comportamiento de las consultas que se realiza. Este trabajo presenta propuestas de algoritmos y t\'ecnicas  que ayudan en el llenado de datos de prueba en una base de datos, tomando en cuenta las referencias compuestas comunes en modelos basados en ER-Idioms, ademas se presenta algoritmos para obtener el orden correcto en que deben ser llenado con datos de prueba las tablas de una base de datos para lo cual se tiene algoritmos que generan datos seg\'un el tipo dato. Para el caso de las llaves foraneas se ofrece mecanismos que autogeneran a partir de la tabla referenciada para evitar la inconsistencia de datos. Otro caso importante es el tipo de dato binario, en este proyecto se opta por trabajar con Postgresql que ofrece su tipo de dato \texttt{bytea} para almacenar archivos en la base de datos como demostraci\'on de todo lo desarrollado se presenta un prototipo que genera datos directamente a la base de datos con la opci\'on de crear un archivo con extension sql siempre que la base de datos no tenga tipo de datos binarios.
