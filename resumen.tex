\chapter*{Resumen}
\thispagestyle{empty}
\renewcommand{\footrulewidth}{0.0pt}% linea footer
En el desarrollo de software que implica el uso de base de datos, es bastante com\'un requerir realizar pruebas para conocer el comportamiento del software conjuntamente con la base de datos, para lo cual es necesario tener la base de datos poblada con datos de prueba similar cuando el software este en un entorno de producci\'on. 

Si bien tenemos claro que es necesario tener la base de datos poblada, no resulta ser tan sencillo poblar con datos de prueba por razones como resulta ser tedioso debido a que no siempre se tiene en mente lo que se quiere, tener en mente nombres de las tablas y quien referencia a quien, por lo cual normalmente se acostumbra a insertar pocos datos datos de prueba, como resultado nos deja la incertedumbre de como vaya ser el comportamiento de las consultas que se realiza. Este trabajo se presenta una propuesta algoritmos y t\'ecnicas  que ayudan de alguna manera el llenado de datos de prueba en una base de datos basados, tomando encuenta las referencias compuestas comunes en modelos ER basados en Idioms, ademas se presenta algoritmos para obtener el orden correcto en que deben ser llenado con datos de prueba las tablas de una base de datos para lo cual se tiene algoritmos que generan datos segun el tipo dato, para el caso de las llaves foraneas se ofrece mecanismos que autogeneran apartir de la tabla al que referencia para evitar la inconsistencia de datos, otro caso importante es el tipo de dato binario en este proyecto se opta por trabajar con Postgresql que ofrece su tipo de dato \texttt{bytea} para almacenar archivos en la base de datos como demostracion de todo lo desarrollado se presenta un prototipo que genera datos directamente a la base de datos con la opcion de crear un archivo con extension sql siempre que la base de datos no tenga tipo de datos binarios.
