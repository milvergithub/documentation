%% ## Construye tu propia portada ##
%% 
%% Una portada se conforma por una secuencia de "Blocks" que incluyen
%% piezas individuales de informaci'on. Un "Block" puede incluir, por
%% ejemplo, el t'itulo del documento, una im'agen (logotipo de la universidad),
%% el nombre del autor, nombre del supervisor, u cualquier otra pieza de
%% informaci'on.
%%
%% Cada "Block" aparece centrado horizontalmente en la p'agina y,
%% verticalmente, todos los "Blocks" se distruyen de manera uniforme 
%% a lo largo de p'agina.
%%
%% Nota tambi'en que, dentro de un mismo "Block" se pueden cortar
%% lineas usando el comando \\
%%
%% El tama'no del texto dentro de un "Block" se puede modificar usando uno de
%% los comandos:
%%   \small      \LARGE
%%   \large      \huge
%%   \Large      \Huge
%%
%% Y el tipo de letra se puede modificar usando:
%%   \bfseries - negritas
%%   \itshape  - it'alicas
%%   \scshape  - small caps
%%   \slshape  - slanted
%%   \sffamily - sans serif
%%
%% Para producir plantillas generales, la informaci'on que ha sido inclu'ida
%% en el archivo principal "tesis.tex" se puede accesar aqu'i usando:
%%   \insertauthor
%%   \inserttitle
%%   \insertsupervisor
%%   \insertinstitution
%%   \insertdegree
%%   \insertfaculty
%%   \insertdepartment
%%   \insertsubmitdate

% caratula --------------------------------------------------------------
\newcommand{\umsslogo}{
      \adjustbox{valign=t}{\includegraphics[scale=0.04]{images/umss}}%
}
\newcommand{\fcytlogo}{%
      \adjustbox{valign=t}{\includegraphics[scale=0.1]{images/fcyt}}%
}

% Carátula:
\begin{titlepage}
\pagenumbering{gobble}
\vspace{-2.7cm}
\begin{tabular}[t]{c p{10cm} c}
    \umsslogo & 
    \begin{center}
    \large{\textsc{Universidad Mayor de San Sim\'on }} \\
    \large{\textsc{Facultad de Ciencias y Tecnolog\'ia }} \\
    \large{\textsc{Carrera de Ingenier\'ia de Sistemas}} 
    \end{center}
    &
    \fcytlogo \\
\end{tabular}

\begin{tabular}{ l l l }
\umsslogo & 
\begin{tabular}[t]{@{}c@{}}UNIVERSIDAD MAYOR DE SAN SIMON\\ FACULTAD DE CIENCIAS Y TECNOLOGIA\\ CARRERA EN INGENIERIA DE SISTEMAS\end{tabular} & 
\fcytlogo \\
\end{tabular}

\vfill

\begin{center}
\huge{\textsc{ALGOR'ITMOS Y T'ECNICAS DE GENERACI'ON DE DATOS DE PRUEBA PARA BASE DE DATOS BASADOS EN IDIOMS}}
\end{center}
\vspace{0.5cm}

%\begin{flushright}
\begin{center}
\textsc{
Proyecto de grado, presentado para optar\\
al Diploma Acad'emico de Licenciatura \\
en Ingenier\'a de Sistemas.
}
%\end{flushright}
\end{center}

\vfill
\begin{tabbing}
\hspace{2cm}\=\+
	\textsc{Presentado por:} Milver Felipe Flores Acevedo	\\
    \\
	\textsc{Tutor:} Lic. Juan Marcelo Flores Acevedo	\\
    \\
	%\textsc{Cochabamba - Bolivia}\\
    \\
\end{tabbing}

\begin{center}
    \textsc{Cochabamba - Bolivia}\\
    \textsc{Agosto - 2016}
\end{center}   

\vfill

%\hrule
%\vspace{0.2cm}
%\noindent\small{Trabajo de Grado \hfill}

\end{titlepage}
%%\setlength{\unitlength}{1 cm} %Especificar unidad de trabajo
%%\thispagestyle{empty}
%%\begin{picture}(18,4)
%%\put(0,0){\includegraphics[width=3cm,height=4cm]{uco.jpg}}
%%\put(11.5,0){\includegraphics[width=4cm,height=4cm]{eupinf.jpg}}
%%\end{picture}
%%\\
%%\\
%%\begin{center}
%%\textbf{{\Huge Universidad de Córdoba}\\[0.5cm]
%%{\LARGE Escuela Politécnica Superior}}\\[1.25cm]
%%{\Large Redes}\\[2.3cm]
%%{\LARGE \textbf{Apuntes}}\\[3.5cm]
%%{\large Tu nombre}\\[2cm]
%%2º Ingeniería Informática (2º Ciclo)\\[1cm]
%%Córdoba - \today
%%\end{center}

%% Nota 1:
%% Se puede agregar un escudo o logotipo en un "Block" como:
%%   \TitleBlock{\includegraphics[height=4cm]{escudo_uni}}
%% y teniendo un archivo "escudo_uni.pdf", "escudo_uni.png" o "escudo_uni.jpg"
%% en alg'un lugar donde LaTeX lo pueda encontrar.

%% Nota 2:
%% Normalmente, el espacio entre "Blocks" se extiende de modo que el
%% contenido se reparte uniformemente sobre toda la p'agina. Este
%% comportamiento se puede modificar para mantener fijo, por ejemplo, el
%% espacio entre un par de "Blocks". Escribiendo:
%%   \TitleBlock{Bloque 1}
%%   \TitleBlock[\bigskip]{Bloque2}
%% se deja un espacio "grande" y de tama~no fijo entre el bloque 1 y 2.
%% Adem'as de \bigskip est'an tambi'en \smallskip y \medskip. Si necesitas
%% aun m'as control puedes usar tambi'en, por ejemplo, \vspace*{2cm}.


