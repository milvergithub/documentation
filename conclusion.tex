\chapter{Conclusiones}
En este trabajo se presenta un dise\~no e implementac\'ion de un conjunto de algoritmos para generar datos de prueba para base de datos y t\'ecnicas para obtener el orden en que se debe llenarlo, con el objetivo de automatizar el proceso del llenado de una base de datos, entre las t\'ecnicas y algoritmos mas importantes se puede mencionar:

\begin{enumerate}
\item La implementac\'ion de algor\'itmos para obtener el orden correcto apartir de una lista de tablas pertenecientes a una base de datos, para lo cual se tom\'o encuenta diferentes casos que podria darse en el dise\~no de una base datos. Como resultado se tiene una lista de tablas ordenadas segun el orden correcto en que deben ser llenados, en el resultado se podria tener un lista de conjuntos de tablas donde las tablas que est\'en en el mismo conjunto tienen el mismo nivel de prioridad.
\item La implemetaci\'on de t\'ecnicas para el manejo referencial de las llaves primarias y foraneas tomando encuenta que una base de datos est\'e  basada en el concepto E-R Idioms. Obteneniendo como resultado el correcto manejo de las llaves foraneas evitando asi la inconsistencia de datos.
\item La implementac\'ion de algoritmos para generar datos con caracteres validos que se asemejen mas a datos reales.
\item La implementac\'ion de un prototipo como parte de la demostrac\'ion de los algoritmos y tecnicas propuestos para lo cual se eligio usar como \texttt{DBMS} PostgreSQL, la razon principal es porque soporta llaves compuestas necesarias para una base de datos basados en E-R Idioms.  
\item La implementac\'ion de t\'ecnicas para insertar un tipo de dato \texttt{bytea} en una base de datos, para lo cual es necesario la convers\'ion previa, como en el proyecto se puede generar datos en el caso \texttt{bytea} es necesario almacenar los archivos en un directorio para luego convertir \texttt{pg\_escape\_bytea(archivo)} e insertar.
\end{enumerate}

Es importante aclarar que al momento de crear un proyecto de configurac\'ion el usuario debe tener previligios de acceso a metadatos del \texttt{DBMS}.