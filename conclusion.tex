\chapter{Conclusiones}
En este trabajo se presenta un dise\~no e implementacion de un conjunto de algoritmos para generar datos de prueba para base de datos y algunas t\'ecnicas para obtener el orden en que se debe llenarlo, con el objetivo automatizar el proceso del llenado de una base de datos, entre las t\'ecnicas y algoritmos mas importantes podemos mencionar.

\begin{enumerate}
\item La implementac\'ion de algor\'itmos para obtener el orden correcto apartir de una lista de tablas pertenecientes a una base de datos, para lo cual se tom\'o encuenta diferentes casos que podria darse en el dise\~no de una base datos. Como resultado se tiene una lista de tablas ordenadas segun el orden correcto en que deben ser llenados, en el resultado se podria tener un lista de conjuntos de tablas donde las tablas que est\'en en el mismo conjunto tienen el mismo nivel de prioridad.
\item La implemetaci\'on de t\'ecnicas para el manejo referencial de las llaves primarias y foraneas tomando encuenta que una base de datos est\'e  basada en el concepto E-R Idioms. Obteneniendo como resultado el correcto manejo de las llaves foraneas evitando asi la inconsistencia de datos.
\item La implementac\'ion de algoritmos para generar palabras, nombres, fechas, correos electronicos.
\item La implementacion de un prototipo   
\end{enumerate}